\begin{thebibliography}{9}
  \bibitem{joy}
    パスカルキッズマガジン,
    "幼児教育に遊びは超重要!プロが幼児期の遊びの大切さを教えます",
    https://www.kaichi-sg.jp/pascal-kids-magazine/education-plays/
  \bibitem{puzzle}
    Huda Fitriyani,Neneng Tasu’ah,
    "The Use of Three Dimensional Puzzle as a Media to Improve Visual-Spatial Intelligence of Children Aged 5-6 Years Old",
    Indonesian Journal of Early Childhood Education Studies(2014)
  \bibitem{rubik}
    公立諏訪東京理科大学篠原研究室,
    "ルービックキューブ学習による脳活動への影響・創造性テスト成績の変化に関する調査",
    https://www.megahouse.co.jp/rubikcube/results/:w
  \bibitem{全天球} 
    瀬戸崎典夫,吉冨諒,岩崎勤,全炳徳,
    "全天球パノラマVRコンテンツを有する平和教育教材の開発",
    日本教育工学会論文誌(2015)
  \bibitem{授業実践}
    瀬戸崎典夫, 森田裕介,竹田仰,
    "ニーズ調査に基づいた多視点型VR教材の開発と授業実践",
    日本バーチャルリアリティ学会論文誌(2006)
  \bibitem{MCT}
    鈴木賢次郎,
    "認知図学事始め-切断面実形視テストによって評価される空間認識力-",
    図学研究,第33巻3号pp.5-12(1999)
  \bibitem{N-back}
    國見充展,松川順子,
    "N-back 課題を用いた視覚的ワーキングメモリの保持と処理の加齢変化",
    心理学研究,第80巻第2号pp.98-104(2009)
\end{thebibliography}
