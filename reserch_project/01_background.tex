\section{背景}
  %======================================================================================
  幼児教育において,パズルやゲームのような「遊び」は
  心身の成長を促すために重要である\cite{joy}.
  %======================================================================================
  Huda Fitriyaniらの研究\cite{puzzle}では5,6歳の児童に3Dパズル活動を実施させた.
  実際に使用されたパズルは家具や建物のミニチュアを模したパズルである.
  児童たちはこれらのパズルをバラバラの状態から,
  正しく組み立てるという活動をおこなった.
  研究の結果,こうしたパズル活動は空間認識や方向感覚の向上に寄与することが判明した.
  さらにルービックキューブのような複雑な立体パズルも,
  認知機能に好影響を与えることが分かっている.
  国立諏訪東京理科大学の研究\cite{rubik}によると,
  ルービックキューブを解くことで
  自頭力と深く関わるとされる前頭前野が活性化することが分かった.
  特に論理的思考力を司る左側の前頭前野に大きく影響することが判明した.
  ルービックキューブ学習後のテストの成績では,
  「応用力」・「思考の速さ」・「思考の深さ」において向上が見られた.
  \\\indent
  %======================================================================================
  このような教育的効果を持つ遊びを,
  より幅広く効果的に提供する手段として近年VR技術が注目されている
  \cite{全天球}\cite{授業実践}.
  VRは仮想空間で様々な現象を再現することが可能で,
  現実の制約にとらわれないオブジェクトや環境を作り出し,
  実際に見ることや体験できないことを立体的に提示することが可能である.
  %======================================================================================
  VR空間上に再現されたルービックキューブを解いた際に,
  現実のルービックキューブを解いた場合と同等の効果を得ることができたならば,
  新たな教材として活用できる可能性がある.
  %======================================================================================
